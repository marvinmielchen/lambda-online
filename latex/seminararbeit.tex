\documentclass{article}
\usepackage{csquotes}

% Packages
\usepackage[german]{babel}

\usepackage[style=ieee]{biblatex}
\usepackage{amsmath}
\usepackage{amssymb}
\usepackage{graphicx}
\usepackage{hyperref} 

\addbibresource{literatur.bib}

% Title page
\title{Implementierung eines Interpreters für eine Programmiersprache auf Basis des untypisierten Lambda-Kalküls unter Verwendung von De-Bruijn-Indizes.}
\author{Marvin Mielchen}
\date{\today}

\begin{document}

\maketitle

\section{Introduction}



- es muss einen Teil in der Arbeit geben der sich mit den Limitierungen der Arbeit beschäftigt
    - probleme mit rekusion in lambo
    - probleme mit speicher in lambo: z.B. sind while schleifen über große arrays wahrscheinlich nicht möglich weil alle expressions ausgewertet werden müssen und
    das also nicht "echt" iterativ ist
    - usw.

    
% BEGIN: Introduction
This is the introduction of your thesis.
% END: Introduction

\section{Background}
% BEGIN: Background
This is the background section of your thesis.
% END: Background

\section{Methodology}
% BEGIN: Methodology
This is the methodology section of your thesis.
% END: Methodology

\section{Results}
% BEGIN: Results
This is the results section of your thesis.
% END: Results

\section{Conclusion}
% BEGIN: Conclusion
This is the conclusion of your thesis.
\cite[S. 8]{nystrom}
% END: Conclusion

\printbibliography
\end{document}
